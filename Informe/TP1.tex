\documentclass[a4paper]{article}
\usepackage[utf8]{inputenc}
\usepackage[spanish, es-tabla]{babel}

\usepackage{amsmath}
\usepackage{amsfonts}
\usepackage{amssymb}

\usepackage{float}
\usepackage{graphicx}
\graphicspath{ {./Imagenes/} }

\usepackage{multirow}
\setlength{\doublerulesep}{\arrayrulewidth}

\usepackage{tikz}
\usetikzlibrary{matrix,calc}

\usepackage{array}
\newcolumntype{C}[1]{>{\centering\let\newline\\\arraybackslash\hspace{0pt}}m{#1}}

\usepackage[american]{circuitikz}

\usepackage{fancyhdr}

\usepackage{units} 

\pagestyle{fancy}
\fancyhf{}
\lhead{22.13 Electrónica III}
\rhead{Mechoulam, Lambertucci, Martorel, Londero}
\rfoot{Página \thepage}

\input{Karnaugh/karnaugh-map.tex}

\begin{document}

%%%%%%%%%%%%%%%%%%%%%%%%%%%%%%%%%%%%%%%%%%%%%%%%%%%%%%%%%%%%%%%%%%%%%%%%% 
%								CARATULA								%
%%%%%%%%%%%%%%%%%%%%%%%%%%%%%%%%%%%%%%%%%%%%%%%%%%%%%%%%%%%%%%%%%%%%%%%%% 

\begin{titlepage}
\newcommand{\HRule}{\rule{\linewidth}{0.5mm}}
\center
\mbox{\textsc{\LARGE \bfseries {Instituto Tecnológico de Buenos Aires}}}\\[1.5cm]
\textsc{\Large 22.13 Electrónica III}\\[0.5cm]


\HRule \\[0.6cm]
{ \Huge \bfseries Trabajo práctico N$^{\circ}$1}\\[0.4cm] 
\HRule \\[1.5cm]


{\large

\emph{Grupo 3}\\
\vspace{3px}

\begin{tabular}{lr} 	
\textsc{Mechoulam}, Alan  &  58438\\
\textsc{Lambertucci}, Guido Enrique  & 58009 \\
\textsc{Martorel}, Ariel  & Legajo \\
\textsc{Londero Bonaparte}, Tomás Guillermo  & 58150 \\
\end{tabular}

\vspace{20px}

\emph{Profesor}\\
\vspace{3px}
\textsc{Dewald}, Kevin\\	

\vspace{100px}

\begin{tabular}{ll}

Presentado: & /19\\

\end{tabular}

}

\vfill

\end{titlepage}


%%%%%%%%%%%%%%%%%%%%%%%%%%%%%%%%%%%%%%%%%%%%%%%%%%%%%%%%%%%%%%%%%%%%%%%%% 
%								INFORME									%
%%%%%%%%%%%%%%%%%%%%%%%%%%%%%%%%%%%%%%%%%%%%%%%%%%%%%%%%%%%%%%%%%%%%%%%%%

\section*{Introducción}


\section*{Desarrollo de la experiencia}


\subsection*{Ejercicio 2}

Dadas las siguientes expresiones:

\begin{equation}
f \left( e,d,c,b,a \right) = \sum m \left( 0,2,4,7,8,10,12,16,18,20,23,24,25,26,27,28 \right)
\label{equ:minterms}
\end{equation}

\begin{equation}
f \left( d,c,b,a \right) = \prod \left( M_0,M_2,M_4,M_7,M_8,M_{10},M_{12} \right)
\label{equ:maxterms}
\end{equation}

se procede a hallar la mínima expresión posible para ambas usando álgebra booleana y mapas de Karnaugh.
Para la expresión (\ref{equ:minterms}):

\begin{center}
\[
	f \left( e,d,c,b,a \right) = \bar{e}\bar{d}\bar{c}\bar{b}\bar{a} \ + \ \bar{e}\bar{d}\bar{c}b\bar{a} \ + \ \bar{e}\bar{d}c\bar{b}\bar{a} \ + \ \bar{e}\bar{d}cba \ + \ \bar{e}d\bar{c}\bar{b}\bar{a} \ + \ \bar{e}d\bar{c}b\bar{a} \ + \ \bar{e}dc\bar{b}\bar{a} \ +
\]
\[
	e\bar{d}\bar{c}\bar{b}\bar{a} \ + \ e\bar{d}\bar{c}b\bar{a} \ + \ e\bar{d}c\bar{b}\bar{a} \ + \ e\bar{d}cba \ + \ ed\bar{c}\bar{b}\bar{a}\ + \ ed\bar{c}\bar{b}a \ + \ ed\bar{c}b\bar{a} \ + \ ed\bar{c}ba \ + \ edc\bar{b}\bar{a} 
\]
El desarrollo utilizando algebra booleana es el siguiente:
\[
	f \left( e,d,c,b,a \right) = \underbrace{\bar{e}\bar{d}\bar{c}\bar{b}\bar{a} \ + \ \bar{e}\bar{d}\bar{c}b\bar{a} }_{\bar{e}\bar{d}\bar{c}\bar{a}}\ + 							\underbrace{\bar{e}\bar{d}\bar{c}\bar{b}\bar{a} \ + \bar{e}\bar{d}c\bar{b}  \bar{a}  }_{\bar{e}\bar{d}\bar{b}\bar{a}}\  +
				\underbrace{\bar{e}\bar{d} c \bar{b}\bar{a} \ + e \bar{d} c\bar{b}  \bar{a}  }_{\bar{d}c\bar{b}\bar{a}}\  +
				\underbrace{\bar{e}\bar{d} c b a \ + e \bar{d} c b a  }_{ \bar{d} c b a}\ +
\]
\[
				\underbrace{\bar{e} d  \bar{c} \bar{b} \bar{a} \ + \bar{e} d \bar{c} b \bar{a}  }_{ \bar{e} d \bar{c}  \bar{a}}\ +
				\underbrace{\bar{e} d  c \bar{b} \bar{a} \ + \bar{e} d \bar{c} \bar{b} \bar{a}  }_{ \bar{e} d \bar{b}  \bar{a}}\ +
				\underbrace{\bar{e} d  c \bar{b} \bar{a} \ + e d c \bar{b} \bar{a}  }_{  d c \bar{b}  \bar{a}}\ +
				\underbrace{e \bar{d} \bar{c} \bar{b} \bar{a} \ + e \bar{d} \bar{c} b \bar{a}  }_{  e\bar{d}  \bar{c}  \bar{a}}\ +
\]
\[
				\underbrace{e d \bar{c} \bar{b} \bar{a} \ + e d \bar{c} \bar{b} a }_{  e d  \bar{c}  \bar{b}}\ +
				\underbrace{e  d \bar{c} \bar{b} a \ + e d \bar{c} b a  }_{  ed  \bar{c}  a}\ +
				\underbrace{\underbrace{e d \bar{c} \bar{b} \bar{a} \ + e d \bar{c} b \bar{a}  }_{  ed  \bar{c}  \bar{a}}\ +
				\underbrace{e d \bar{c} \bar{b} a \ + e d \bar{c} b a  }_{ ed \bar{c}  \bar{a}}}_{  ed  \bar{c}  \bar{a}} =
\]
De la anterior expresion, reordenando se consigue:
\[
				f \left( e,d,c,b,a \right) =\underbrace{\bar{e} \bar{d} \bar{c} \bar{b} \bar{a} \ + e \bar{d} \bar{c} \bar{a}  }_{  \bar{d}  \bar{c}  \bar{a}}\ +
						\underbrace{\bar{e} \bar{d} \bar{b} \bar{a}  \ + \bar{e} \bar{d} \bar{b} \bar{a}  }_{  \bar{d}  c b a}\ +
						\bar{d} cba + ed\bar{c}\bar{b}+
						\underbrace{\bar{d} c \bar{b} \bar{a} \ + d c \bar{b} \bar{a}  }_{  c \bar{b} \bar{a}}\ +
\]
\[	
					\underbrace{e d  \bar{c} \bar{a} \ + e d \bar{c} a  }_{  ed \bar{c} }\ +
					\underbrace{\bar{e} d \bar{c} \bar{a} \ + e d \bar{c} \bar{a}  }_{  d \bar{c} \bar{a}}
\]
\[	
				f \left( e,d,c,b,a \right) =\underbrace{\bar{d}   \bar{c} \bar{a} \ + d  \bar{c} \bar{a}   }_{  \bar{c}\bar{a} }\ +
						\underbrace{e  d \bar{c} \ +e d  \bar{c} \bar{b}   }_{  e  d \bar{c}  }\ +\bar{d}cba+\bar{e}\bar{b}\bar{a}+c\bar{b}\bar{a}
\]
teniendo en cuenta que 

\[
		c\bar{b}\bar{a}= 
\]
\[
		\bar{e}\bar{b}\bar{a}=
\]

se llega a la expresión

\[
		f \left( e,d,c,b,a \right) =bac\bar{d}+ed\bar{c}+\bar{c}\bar{a}+\bar{b}\bar{a}
\]
\end{center}

\begin{centering}
    \begin{Karnaugh}
        \minterms{0,2,4,7,8,10,12}
        \maxterms{1,3,5,6,9,11,13,14,15}
        
        \implicant{0}{8}{green}
        \implicantsol{7}{red}

        \implicantcantons{blue}
        
    \end{Karnaugh}
\par\end{centering}

\begin{center}
\textbf{e = 0}
\end{center}

\begin{centering}
    \begin{Karnaugh}
 \minterms{0,2,4,7,8,9,10,11,12}
        \maxterms{1,3,5,6,13,14,15}
        
        \implicant{0}{8}{green}
        \implicantsol{7}{red}
        \implicant{8}{10}{orange}

        \implicantcantons{blue}
        
    \end{Karnaugh}
\par\end{centering}

\begin{center}
\textbf{e = 1}
\end{center}

\begin{table}[H]
\centering
\caption{Mapa de Karnaugh de la expresión (\ref{equ:minterms}).}
\label{tabla:maxterms}
\end{table}



En esta se pueden observar 4 grupos distintos:
\begin{enumerate}
	\item Compuesto por los casilleros 0, 4, 8, 12, 16, 20, 24 y 28, obteniéndose la expresión $ b a \bar{d} c $;
	\item Compuesto por los casilleros 7 y 23, obteniéndose la expresión $ e d \bar{c} $;
	\item Compuesto por los casilleros 0, 2, 8, 10, 16, 18, 24 y 26, obteniéndose la expresión $ \bar{c} \bar{a} $;
	\item Compuesto por los casilleros 24, 25, 26 y 27, obteniéndose la expresión $ \bar{b} \bar{a} $
\end{enumerate}

de esta forma se llega a:
\begin{equation}
	f \left( e,d,c,b,a \right) = b a \bar{d} c \ + \  e d \bar{c} \ + \ \bar{c} \bar{a} \ + \ \bar{b} \bar{a}
\end{equation}

\begin{figure}[H]
	\centering
	\includegraphics[width=0.9\textwidth]{Circuito1.PNG}
\caption{Circuito resultante de simplificar la expresión (\ref{equ:minterms}).}
	\label{fig:circ1}
\end{figure}

Por otro lado, para la expresión (\ref{equ:maxterms}) se escribe en forma de maxterminos:

\[
	f \left( d,c,b,a \right) = \left( a + b + c + d \right) \cdot \left( a + \bar{b} + c + d\right) \cdot \left( a + b+ \bar{c} + d \right) \cdot \left( \bar{a} + \bar{b} + \bar{c} + d \right) \cdot
\]
\[
	\left( a + b + c + \bar{d} \right) \cdot \left( a + \bar{b} + c + \bar{d} \right) \cdot \left( a + b + \bar{c} +\bar{d} \right)
\]

\begin{centering}
    \begin{Karnaugh}
        \minterms{1,3,5,6,9,11,13,14,15}
        \maxterms{0,2,4,7,8,10,12}
        
        \implicant{1}{9}{yellow}
        \implicant{6}{14}{blue}
        \implicant{13}{11}{red}
        \implicantdaltbaix{1}{11}{green}
        
    \end{Karnaugh}
\par\end{centering}


\begin{table}[H]
\centering
\caption{Mapa de Karnaugh de la expresión (\ref{equ:maxterms}).}
\label{tabla:maxterms}
\end{table}

En esta se pueden observar 3 grupos:
\begin{enumerate}
	\item Compuesto por los casilleros 0, 4, 8 y 12, obteniéndose la expresión $ b + a $;
	\item Compuesto por el casillero 7, obteniéndose la expresión $ \bar{a} + \bar{b} + \bar{c} + d $;
	\item Compuesto por los casilleros 0, 2, 8 y 10, obteniéndose la expresión $ c + a $
\end{enumerate}

obteniendo finalmente la expresión: 

\begin{equation}
	f \left( d,c,b,a \right) = \left( b \ + \ a \right) \cdot \left( c \ + \ a \right) \cdot \left( \bar{a} \ + \ \bar{b} \ + \ \bar{c} \ + \ d \right)
\end{equation}

\begin{figure}[H]
	\centering
	\includegraphics[width=0.9\textwidth]{Circuito2.PNG}
\caption{Circuito resultante de simplificar la expresión (\ref{equ:maxterms}).}
	\label{fig:circ2}
\end{figure}


\section*{Conclusión}


\end{document}
