\documentclass[a4paper]{article}
\usepackage[utf8]{inputenc}
\usepackage[spanish, es-tabla]{babel}

\usepackage{amsmath}
\usepackage{amsfonts}
\usepackage{amssymb}

\usepackage{float}
\usepackage{graphicx}
\graphicspath{ {./Imagenes/} }

\usepackage{multirow}
\setlength{\doublerulesep}{\arrayrulewidth}

\usepackage{tikz}
\usetikzlibrary{matrix,calc}

\usepackage{array}
\newcolumntype{C}[1]{>{\centering\let\newline\\\arraybackslash\hspace{0pt}}m{#1}}

\usepackage[american]{circuitikz}

\usepackage{fancyhdr}

\usepackage{units} 

\pagestyle{fancy}
\fancyhf{}
\lhead{22.13 Electrónica III}
\rhead{Mechoulam, Lambertucci, Martorel, Londero}
\rfoot{Página \thepage}

\input{Karnaugh/karnaugh-map.tex}

\begin{document}

%%%%%%%%%%%%%%%%%%%%%%%%%%%%%%%%%%%%%%%%%%%%%%%%%%%%%%%%%%%%%%%%%%%%%%%%% 
%								CARATULA								%
%%%%%%%%%%%%%%%%%%%%%%%%%%%%%%%%%%%%%%%%%%%%%%%%%%%%%%%%%%%%%%%%%%%%%%%%% 

\begin{titlepage}
\newcommand{\HRule}{\rule{\linewidth}{0.5mm}}
\center
\mbox{\textsc{\LARGE \bfseries {Instituto Tecnológico de Buenos Aires}}}\\[1.5cm]
\textsc{\Large 22.13 Electrónica III}\\[0.5cm]


\HRule \\[0.6cm]
{ \Huge \bfseries Trabajo práctico N$^{\circ}$1}\\[0.4cm] 
\HRule \\[1.5cm]


{\large

\emph{Grupo 3}\\
\vspace{3px}

\begin{tabular}{lr} 	
\textsc{Mechoulam}, Alan  &  58438\\
\textsc{Lambertucci}, Guido Enrique  & 58009 \\
\textsc{Martorel}, Ariel  & Legajo \\
\textsc{Londero Bonaparte}, Tomás Guillermo  & 58150 \\
\end{tabular}

\vspace{20px}

\emph{Profesor}\\
\vspace{3px}
\textsc{Dewald}, Kevin\\	

\vspace{100px}

\begin{tabular}{ll}

Presentado: & /19\\

\end{tabular}

}

\vfill

\end{titlepage}


%%%%%%%%%%%%%%%%%%%%%%%%%%%%%%%%%%%%%%%%%%%%%%%%%%%%%%%%%%%%%%%%%%%%%%%%% 
%								INFORME									%
%%%%%%%%%%%%%%%%%%%%%%%%%%%%%%%%%%%%%%%%%%%%%%%%%%%%%%%%%%%%%%%%%%%%%%%%%

\section*{Introducción}


\section*{Desarrollo de la experiencia}

\subsection*{Ejercicio 1}
Se realizo un programa que calcula el rango y resolucion de la a entrada
\subsection*{Ejercicio 2}

Dadas las siguientes expresiones:

\begin{equation}
f \left( e,d,c,b,a \right) = \sum m \left( 0,2,4,7,8,10,12,16,18,20,23,24,25,26,27,28 \right)
\label{equ:minterms}
\end{equation}

\begin{equation}
f \left( d,c,b,a \right) = \prod \left( M_0,M_2,M_4,M_7,M_8,M_{10},M_{12} \right)
\label{equ:maxterms}
\end{equation}

se procede a hallar la mínima expresión posible para ambas usando álgebra booleana y mapas de Karnaugh.
Escribiendo la expresión (\ref{equ:minterms}) en forma de minterminos se obtiene:

\begin{center}
\[
	f \left( e,d,c,b,a \right) = \bar{e}\bar{d}\bar{c}\bar{b}\bar{a} \ + \ \bar{e}\bar{d}\bar{c}b\bar{a} \ + \ \bar{e}\bar{d}c\bar{b}\bar{a} \ + \ \bar{e}\bar{d}cba \ + \ \bar{e}d\bar{c}\bar{b}\bar{a} \ + \ \bar{e}d\bar{c}b\bar{a} \ + \ \bar{e}dc\bar{b}\bar{a} \ +
\]
\[
	e\bar{d}\bar{c}\bar{b}\bar{a} \ + \ e\bar{d}\bar{c}b\bar{a} \ + \ e\bar{d}c\bar{b}\bar{a} \ + \ e\bar{d}cba \ + \ ed\bar{c}\bar{b}\bar{a}\ + \ ed\bar{c}\bar{b}a \ + \ ed\bar{c}b\bar{a} \ + \ ed\bar{c}ba \ + \ edc\bar{b}\bar{a} 
\]
Su desarrollo utilizando álgebra booleana es el siguiente:
\[
	f \left( e,d,c,b,a \right) = \underbrace{\bar{e}\bar{d}\bar{c}\bar{b}\bar{a} \ + \ \bar{e}\bar{d}\bar{c}b\bar{a} }_{\bar{e}\bar{d}\bar{c}\bar{a}}\ + 							\underbrace{\bar{e}\bar{d}\bar{c}\bar{b}\bar{a} \ + \bar{e}\bar{d}c\bar{b}  \bar{a}  }_{\bar{e}\bar{d}\bar{b}\bar{a}}\  +
				\underbrace{\bar{e}\bar{d} c \bar{b}\bar{a} \ + e \bar{d} c\bar{b}  \bar{a}  }_{\bar{d}c\bar{b}\bar{a}}\  +
				\underbrace{\bar{e}\bar{d} c b a \ + e \bar{d} c b a  }_{ \bar{d} c b a}\ +
\]
\[
				\underbrace{\bar{e} d  \bar{c} \bar{b} \bar{a} \ + \bar{e} d \bar{c} b \bar{a}  }_{ \bar{e} d \bar{c}  \bar{a}}\ +
				\underbrace{\bar{e} d  c \bar{b} \bar{a} \ + \bar{e} d \bar{c} \bar{b} \bar{a}  }_{ \bar{e} d \bar{b}  \bar{a}}\ +
				\underbrace{\bar{e} d  c \bar{b} \bar{a} \ + e d c \bar{b} \bar{a}  }_{  d c \bar{b}  \bar{a}}\ +
				\underbrace{e \bar{d} \bar{c} \bar{b} \bar{a} \ + e \bar{d} \bar{c} b \bar{a}  }_{  e\bar{d}  \bar{c}  \bar{a}}\ +
\]
\[
				\underbrace{e d \bar{c} \bar{b} \bar{a} \ + e d \bar{c} \bar{b} a }_{  e d  \bar{c}  \bar{b}}\ +
				\underbrace{e  d \bar{c} \bar{b} a \ + e d \bar{c} b a  }_{  ed  \bar{c}  a}\ +
				\underbrace{\underbrace{e d \bar{c} \bar{b} \bar{a} \ + e d \bar{c} b \bar{a}  }_{  ed  \bar{c}  \bar{a}}\ +
				\underbrace{e d \bar{c} \bar{b} a \ + e d \bar{c} b a  }_{ ed \bar{c}  \bar{a}}}_{  ed  \bar{c}  \bar{a}} =
\]
De la anterior expresión, reordenando se consigue:
\[
				f \left( e,d,c,b,a \right) =\underbrace{\bar{e} \bar{d} \bar{c} \bar{b} \bar{a} \ + e \bar{d} \bar{c} \bar{a}  }_{  \bar{d}  \bar{c}  \bar{a}}\ +
						\underbrace{\bar{e} \bar{d} \bar{b} \bar{a}  \ + \bar{e} \bar{d} \bar{b} \bar{a}  }_{  \bar{d}  c b a}\ +
						\bar{d} cba + ed\bar{c}\bar{b}+
						\underbrace{\bar{d} c \bar{b} \bar{a} \ + d c \bar{b} \bar{a}  }_{  c \bar{b} \bar{a}}\ +
\]
\[	
					\underbrace{e d  \bar{c} \bar{a} \ + e d \bar{c} a  }_{  ed \bar{c} }\ +
					\underbrace{\bar{e} d \bar{c} \bar{a} \ + e d \bar{c} \bar{a}  }_{  d \bar{c} \bar{a}}
\]
\[	
				f \left( e,d,c,b,a \right) =\underbrace{\bar{d}   \bar{c} \bar{a} \ + d  \bar{c} \bar{a}   }_{  \bar{c}\bar{a} }\ +
						\underbrace{e  d \bar{c} \ +e d  \bar{c} \bar{b}   }_{  e  d \bar{c}  }\ +\bar{d}cba+\bar{e}\bar{b}\bar{a}+c\bar{b}\bar{a}
\]
teniendo en cuenta que 

\[
		c\bar{b}\bar{a}= ec\bar{b}\bar{a}+\bar{e}c\bar{b}\bar{a}+\bar{c}\bar{e}\bar{b}\bar{a}
\]
\[
		\bar{e}\bar{b}\bar{a}=c\bar{e}\bar{b}\bar{a}+\bar{c}\bar{e}\bar{b}\bar{a}+ce\bar{b}\bar{a}
\]
\[
	\underbrace{\underbrace{\bar{e}\bar{b}\bar{a} \ + ce\bar{b}\bar{a}  }_{ \bar{b}\bar{a}}\ 
	\underbrace{c\bar{b}\bar{a} \ + ce\bar{b}\bar{a}  }_{ \bar{b}\bar{a}}\ }_{\bar{b}\bar{a}}
\]
se llega a la expresión

\begin{equation}
		f \left( e,d,c,b,a \right) =bac\bar{d}+ed\bar{c}+\bar{c}\bar{a}+\bar{b}\bar{a}
		\label{equ:boolmin}
\end{equation}
\end{center}

Por otro lado, utilizando mapas de Karnaugh se consigue el siguiente gráfico:

\begin{centering}
    \begin{Karnaugh}
        \minterms{0,2,4,7,8,10,12}
        \maxterms{1,3,5,6,9,11,13,14,15}
        
        \implicant{0}{8}{green}
        \implicantsol{7}{red}

        \implicantcantons{blue}
        
    \end{Karnaugh}
\par\end{centering}

\begin{center}
\textbf{e = 0}
\end{center}

\begin{centering}
    \begin{Karnaugh}
 \minterms{0,2,4,7,8,9,10,11,12}
        \maxterms{1,3,5,6,13,14,15}
        
        \implicant{0}{8}{green}
        \implicantsol{7}{red}
        \implicant{8}{10}{orange}

        \implicantcantons{blue}
        
    \end{Karnaugh}
\par\end{centering}

\begin{center}
\textbf{e = 1}
\end{center}

\begin{table}[H]
\centering
\caption{Mapa de Karnaugh de la expresión (\ref{equ:minterms}).}
\label{tabla:maxterms}
\end{table}



En este se pueden observar 4 grupos distintos:
\begin{enumerate}
	\item Compuesto por los casilleros 0, 4, 8, 12, 16, 20, 24 y 28, obteniéndose la expresión $ b a \bar{d} c $;
	\item Compuesto por los casilleros 7 y 23, obteniéndose la expresión $ e d \bar{c} $;
	\item Compuesto por los casilleros 0, 2, 8, 10, 16, 18, 24 y 26, obteniéndose la expresión $ \bar{c} \bar{a} $;
	\item Compuesto por los casilleros 24, 25, 26 y 27, obteniéndose la expresión $ \bar{b} \bar{a} $
\end{enumerate}

de esta forma se llega a la expresión:
\[
	f \left( e,d,c,b,a \right) = b a \bar{d} c \ + \  e d \bar{c} \ + \ \bar{c} \bar{a} \ + \ \bar{b} \bar{a}
\]

la cual coincide con la ecuación (\ref{equ:boolmin}). De esta forma se representa, mediante un circuito de compuertas lógicas, la formula hallada.

\begin{figure}[H]
	\centering
	\includegraphics[width=0.9\textwidth]{Circuito1.PNG}
\caption{Circuito resultante de simplificar la expresión (\ref{equ:minterms}).}
	\label{fig:circ1}
\end{figure}

\begin{center}
Por otro lado, la expresión (\ref{equ:maxterms}) se escribe en forma de maxterminos:
\[
	f \left( d,c,b,a \right) = \left( a + b + c + d \right) \cdot \left( a + \bar{b} + c + d\right) \cdot \left( a + b+ \bar{c} + d \right) \cdot \left( \bar{a} + \bar{b} + \bar{c} + d \right) \cdot
\]
\[
	\left( a + b + c + \bar{d} \right) \cdot \left( a + \bar{b} + c + \bar{d} \right) \cdot \left( a + b + \bar{c} +\bar{d} \right)
\]

Su desarrollo utilizando álgebra booleana es el siguiente:
\[
	f \left( d,c,b,a \right) = \underbrace{\left( a + b + c + d \right) \cdot \left( a + b + \bar{c} +\bar{d} \right)}_{a + b} \cdot 
	\left( \bar{a} + \bar{b} + \bar{c} + d \right) \cdot 
\]
\[
	\left( a + b + \bar{c} + d \right) \cdot
	\left( a + b + c + \bar{d} \right) \cdot 
	\left( a + \bar{b} + c + d\right) \cdot 
	\left( a + \bar{b} + c + \bar{d} \right)
\]
\[
	f \left( d,c,b,a \right) = \left( a + b \right) \cdot \left( \bar{a} + \bar{b} + \bar{c} + d \right) \cdot 
	\left( a + \bar{b} + c + d\right) \cdot \left( a + b + \bar{c} + d \right) \cdot
\]
\[
	\left( a + b + c + \bar{d} \right) \cdot \underbrace{\left( a + \bar{b} + c + \bar{d} \right)  \cdot \left( a + b + c + d \right)}_{a + c}
\]

\[
	f \left( d,c,b,a \right) = \left( a + b \right) \cdot \left( \bar{a} + \bar{b} + \bar{c} + d \right) \cdot 
	\left( a + c \right) \cdot
\]
\[
	\underbrace{\left( a + \bar{b} + c + d\right) \cdot \left( a + b + c + \bar{d} \right)\cdot}_{\alpha}
	\underbrace{\left( a + b + \bar{c} + d \right) \cdot \left( a + b + c + \bar{d} \right)}_{\beta}
\]

\textsc{Cálculo auxiliar}
\[
	\alpha = \left[ \left( a + c + d \right) + \bar{b} \right] \cdot 
	\left[ \left( a + c + \bar{d} \right) + b \right]  
\]
\[
	\alpha = \underbrace{\left( a + c + d \right) \cdot \left( a + c + \bar{d} \right)}_{a + c}
	+ \left( a + c + d \right) \cdot b + \left( a + c + \bar{d} \right) \cdot \bar{b} +
	\underbrace{d \bar{d}}_{0}	
\]
\[
	\alpha = a + c + \underbrace{\left( a + c \right) \cdot b + \left( a + c \right) \cdot \bar{b}}_{a + c} 
	+ \underbrace{db + \bar{d}\bar{b}}_{1}	= a + c + 1
\]

De forma análoga se puede llegar a que
\[	\beta = a + b + 1 \]

Por lo tanto,
\[
	\alpha \cdot \beta = \left( a + c + 1 \right) \cdot \left( a + b + 1 \right)
\]

\[
	\alpha \cdot \beta = a + b + c + ab + ac + cb + 1 = 1
\]

Luego,
\begin{equation}
f \left( d,c,b,a \right) = \left( a + b \right) \cdot \left( \bar{a} + \bar{b} + \bar{c} + d \right) \cdot 
	\left( a + c \right)
	\label{equ:boolmax}
\end{equation}

A su vez, usando mapas de Karnaugh, se obtiene lo siguiente:

\end{center}
\begin{centering}
    \begin{Karnaugh}
        \minterms{1,3,5,6,9,11,13,14,15}
        \maxterms{0,2,4,7,8,10,12}
        
        \implicant{1}{9}{yellow}
        \implicant{6}{14}{blue}
        \implicant{13}{11}{red}
        \implicantdaltbaix{1}{11}{green}
        
    \end{Karnaugh}
\par\end{centering}


\begin{table}[H]
\centering
\caption{Mapa de Karnaugh de la expresión (\ref{equ:maxterms}).}
\label{tabla:maxterms}
\end{table}

En esta se pueden observar 3 grupos:
\begin{enumerate}
	\item Compuesto por los casilleros 0, 4, 8 y 12, obteniéndose la expresión $ b + a $;
	\item Compuesto por el casillero 7, obteniéndose la expresión $ \bar{a} + \bar{b} + \bar{c} + d $;
	\item Compuesto por los casilleros 0, 2, 8 y 10, obteniéndose la expresión $ c + a $
\end{enumerate}

obteniendo finalmente la expresión: 

\begin{equation}
	f \left( d,c,b,a \right) = \left( b \ + \ a \right) \cdot \left( c \ + \ a \right) \cdot \left( \bar{a} \ + \ \bar{b} \ + \ \bar{c} \ + \ d \right)
\end{equation}

coincidente con la ecuación	(\ref{equ:boolmax}). Por último, se utiliza dicha formula dicha para elaborar un circuito de compuertas lógicas que la represente.

\begin{figure}[H]
	\centering
	\includegraphics[width=0.9\textwidth]{Circuito2.PNG}
\caption{Circuito resultante de simplificar la expresión (\ref{equ:maxterms}).}
	\label{fig:circ2}
\end{figure}
\subsection*{Ejercicio 3}
\subsection*{Ejercicio 4}
En este punto se pidio armar un circuito que dados 4 bits de entrada en binario lo transforme a codigo de Gray  se armo la tabla de verdad 
\begin{table}[H]
\centering
\begin{tabular}{|c|c|c|c|c|c|c|c|c|}
\hline
\textbf{$d$} & \textbf{$c$} & \textbf{$b$} & \textbf{$a$} & \textbf{$m_{ij}$} & \textbf{$y_4$} & \textbf{$y_3$} & \textbf{$y_2$} & \textbf{$y_1$} \\ \hline
0              & 0              & 0              & 0              & \textbf{$m_i0$}   & 0              & 0              & 0              & 0              \\ \hline
0              & 0              & 0              & 1              & \textbf{$m_i1$}   & 0              & 0              & 0              & 1              \\ \hline
0              & 0              & 1              & 0              & \textbf{$m_i2$}   & 0              & 0              & 1              & 1              \\ \hline
0              & 0              & 1              & 1              & \textbf{$m_i3$}   & 0              & 0              & 1              & 0              \\ \hline
0              & 1              & 0              & 0              & \textbf{$m_i4$}   & 0              & 1              & 1              & 0              \\ \hline
0              & 1              & 0              & 1              & \textbf{$m_i5$}   & 0              & 1              & 1              & 1              \\ \hline
0              & 1              & 1              & 0              & \textbf{$m_i6$}   & 0              & 1              & 0              & 1              \\ \hline
0              & 1              & 1              & 1              & \textbf{$m_i7$}   & 0              & 1              & 0              & 0              \\ \hline
1              & 0              & 0              & 0              & \textbf{$m_i8$}   & 1              & 1              & 0              & 0              \\ \hline
1              & 0              & 0              & 1              & \textbf{$m_i9$}   & 1              & 1              & 0              & 1              \\ \hline
1              & 0              & 1              & 0              & \textbf{$m_iA$}   & 1              & 1              & 1              & 1              \\ \hline
1              & 0              & 1              & 1              & \textbf{$m_iB$}   & 1              & 1              & 1              & 0              \\ \hline
1              & 1              & 0              & 0              & \textbf{$m_iC$}   & 1              & 0              & 1              & 0              \\ \hline
1              & 1              & 0              & 1              & \textbf{$m_iD$}   & 1              & 0              & 1              & 1              \\ \hline
1              & 1              & 1              & 0              & \textbf{$m_iE$}   & 1              & 0              & 0              & 1              \\ \hline
1              & 1              & 1              & 1              & \textbf{$m_iF$}   & 1              & 0              & 0              & 0              \\ \hline
\end{tabular}
\end{table}

Se procede a escribir cada bit de salida en funcion de los minterminos:
\[
	y_4 = \sum_{j=8}^{F} m_{4j}  \  ; \ y_3 = \sum_{j=4}^{B} m_{3j}\  ; \ y_2 = \sum_{j=2}^{5} m_{2j} \ + \  \sum_{j=A}^{D} m_{2j}  \  ; \  y_1=m_{11}+m_{12}+m_{15}+m_{16}+m_{19}+m_{1A}+m_{1D}+m_{1E} 
\]
luego para llegar a la forma simplificada se hizo el mapa de Karnaugh de cada salida:
\begin{centering}
    \begin{Karnaugh}
        \minterms{8,9,10,11,12,13,14,15}
        \maxterms{0,1,2,3,4,5,6,7}
        
        \implicant{12}{10}{red}
   
    \end{Karnaugh}
\par\end{centering}


\begin{table}[H]
\centering
\caption{Mapa de Karnaugh del bit y4 de salida.}
\label{tabla:maxterms}
\end{table}
Se puede ver que $$y_4 = d$$
del segundo bit:
\begin{centering}
    \begin{Karnaugh}
        \minterms{4,5,6,7,8,9,10,11}
        \maxterms{0,1,2,3,12,13,14,15}
        
        \implicant{4}{6}{red}
        \implicant{8}{10}{red}
    \end{Karnaugh}
\par\end{centering}


\begin{table}[H]
\centering
\caption{Mapa de Karnaugh del  bit y3 de salida.}
\label{tabla:maxterms}
\end{table}
De aqui $$y_3 = \bar{d}c+d\bar{c}$$
continuando para la siguiente salida:
\begin{centering}
    \begin{Karnaugh}
        \minterms{2,3,4,5,10,11,12,13}
        \maxterms{0,1,6,7,8,9,14,15}
        
        \implicant{4}{13}{red}
\implicantdaltbaix[3pt]{3}{10}{red}
    \end{Karnaugh}
\par\end{centering}


\begin{table}[H]
\centering
\caption{Mapa de Karnaugh del bit y2 de salida.}
\label{tabla:maxterms}
\end{table}
De aqui $$y_2 = \bar{c}b+c\bar{b}$$
continuando para la ultima salida:
\begin{centering}
    \begin{Karnaugh}
        \minterms{1,2,5,6,9,10,13,14}
        \maxterms{0,3,4,7,8,11,12,15}        
        \implicant{1}{9}{red}
        \implicant{2}{10}{red}
    \end{Karnaugh}
\par\end{centering}


\begin{table}[H]
\centering
\caption{Mapa de Karnaugh del  bit y1 de salida.}
\label{tabla:maxterms}
\end{table}
Finalmente se obtiene: $$y_1 = \bar{b}a+\bar{a}b$$
Luego se armo el circuito únicamente utilizando compuertas \textbf{OR AND } y \textbf{NOT}

\begin{figure}[H]
	\centering
	\includegraphics[width=0.9\textwidth]{Circuito3.PNG}
	\label{fig:circ3}
\end{figure}
\section*{Conclusión}
Hablar algo del codigo de verilog
\subsection*{Ejercicio 5}
\end{document}
